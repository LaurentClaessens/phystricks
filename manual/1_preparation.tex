%+++++++++++++++++++++++++++++++++++++++++++++++++++++++++++++++++++++++++++++++++++++++++++++++++++++++++++++++++++++++++++ 
\section{Preparation}
%+++++++++++++++++++++++++++++++++++++++++++++++++++++++++++++++++++++++++++++++++++++++++++++++++++++++++++++++++++++++++++

%--------------------------------------------------------------------------------------------------------------------------- 
\subsection{Dependencies and installation}
%---------------------------------------------------------------------------------------------------------------------------

\begin{enumerate}
    \item
        You need a working \href{ http://sagemath.org }{ sage } installation.
    \item
        Download \phystricks\ from \href{ https://github.com/LaurentClaessens/phystricks }{ github } and make it available from Sage (\info{from phystricks import *} has to work).
    \item
        I don't even speak about having a working \LaTeX\ installation with Tikz installed.
\end{enumerate}

%--------------------------------------------------------------------------------------------------------------------------- 
\subsection{In your \LaTeX\ file}
%---------------------------------------------------------------------------------------------------------------------------

The preamble of your \LaTeX\ file has to contain

\begin{verbatim}
    \usepackage{calc}   
    \usepackage{tikz}
    \usetikzlibrary{patterns}
    \usetikzlibrary{calc}
    \newcounter{defHatch}
    \newcounter{defPattern}
    \setcounter{defHatch}{0}
    \setcounter{defPattern}{0}
\end{verbatim}

and you (don't really) have to compile with \info{pdflatex -shell-escape}.

%--------------------------------------------------------------------------------------------------------------------------- 
\subsection{Where do I find examples ?}
%---------------------------------------------------------------------------------------------------------------------------

You will found (figuratively) tons of examples in the following documents :
\begin{enumerate}
    \item
        In the demo document. The sources are included in the \phystricks's repository; in the subdirectory \info{phystricks/testing/demonstration}. Browse the pdf at \url{http://laurent.claessens-donadello.eu/pdf/phystricks-demo.pdf}.
    \item
        In mazhe. Download the source at \url{ https://github.com/LaurentClaessens/mazhe/ } and browse the pdf at \url{http://laurent.claessens-donadello.eu/pdf/mazhe.pdf}.
    \item
        In smath.Download the source at \url{ https://github.com/LaurentClaessens/smath/ } and browse the pdf at \url{http://laurent.claessens-donadello.eu/pdf/smath.pdf}.
\end{enumerate}
Since every single functionality of \phystricks\ is used in at least one picture of \info{mazhe} or \info{smath}, we are not going to give so much examples in this document.

You are also invited to read the file \info{Constructors.py}; the docstring are explaining the creation of most of the graph types.

If you need something special or if you encounter any difficulty, send me an email.

%--------------------------------------------------------------------------------------------------------------------------- 
\subsection{Structure of your \phystricks\ file}
%---------------------------------------------------------------------------------------------------------------------------

Most of your \phystricks files will have the following structure :

\lstinputlisting{phystricksQLXFooBDalHMaT.py}

We will see later the significance of these lines.

%+++++++++++++++++++++++++++++++++++++++++++++++++++++++++++++++++++++++++++++++++++++++++++++++++++++++++++++++++++++++++++
\section{Draw points}
%+++++++++++++++++++++++++++++++++++++++++++++++++++++++++++++++++++++++++++++++++++++++++++++++++++++++++++++++++++++++++++

Here is the code corresponding to one red point with two marks.

\lstinputlisting{phystricksOnePoint.py}

\begin{enumerate}
    \item
        
Compile it once in the Sage terminal :

\lstinputlisting{sageSnip_1.py}

\item

    As suggested by the Sage's output input the file \info{Fig\_OnePoint.pstricks} in your \LaTeX\ document.

\item

    Compile your document with \info{pdflatex <mydocument> -shell-escape}

\item
    Re-do the compilation in Sage
\item
    Re-do the \LaTeX\ compilation.
\end{enumerate}
If you don't compile twice, some elements can be badly placed, especially the marks that you put on points like the \( P\) and \( Q\) in this example.

If you want to know why, this is related to the mechanism of catching the \LaTeX's internal counters(here the size of the box) by \phystricks, see section \ref{SECooKVXMooMKJAXV}.

The result should be

\begin{center}
   \input{Fig_OnePoint.pstricks}
\end{center}

%--------------------------------------------------------------------------------------------------------------------------- 
\subsection{Segments}
%---------------------------------------------------------------------------------------------------------------------------

%///////////////////////////////////////////////////////////////////////////////////////////////////////////////////////////
\subsubsection{Orthogonal}
%///////////////////////////////////////////////////////////////////////////////////////////////////////////////////////////

If \info{seg} is a segment from \info{A} to \info{B}. There are many cases in which you want a segment orthogonal to \info{seg}.

\begin{enumerate}
    \item
        Let \info{P} be a point outside \info{seg}. The segment from \info{P} to its orthogonal projection on \info{seg} is

        \lstinputlisting{sageSnip_12.py}
    \item
        Let \info{P} be a point on \info{seg}. The segment from \info{P} which is orthogonal to \info{seg} is

        \lstinputlisting{sageSnip_13.py}

        If you do not provide the optional argument \info{point}, it will be the initial point of \info{seg}.
\end{enumerate}

%+++++++++++++++++++++++++++++++++++++++++++++++++++++++++++++++++++++++++++++++++++++++++++++++++++++++++++++++++++++++++++ 
\section{Drawing curves}
%+++++++++++++++++++++++++++++++++++++++++++++++++++++++++++++++++++++++++++++++++++++++++++++++++++++++++++++++++++++++++++

All the curves are internally converted into parametric curve and then transformed into a large number of small segments. Tikz will only see these segments. For that reason, we are able to draw virtually anything that Sage can compute : we are not bound by Tikz's internals, and even less by \LaTeX's \sout{legacyembarrassingmakemecrazy} limitations.

%--------------------------------------------------------------------------------------------------------------------------- 
\subsection{Drawing functions}
%---------------------------------------------------------------------------------------------------------------------------

For drawing the function \( x\mapsto x^2\) on \( \mathopen[ mx , Mx \mathclose]\) the syntax is :

\lstinputlisting{sageSnip_3.py}

The function itself (what is inside the \info{phyFunction} argument) is a Sage expression, so respecting the Sage syntax and using any function that Sage know.

In fact you can put inside \info{phyFunction} (I guess) anything that has a \info{\_\_call\_\_} method, as long as it returns real numbers.

The following is legal:
\lstinputlisting{sageSnip_4.py}
and draws the graph of
\begin{equation}
    x\mapsto \int_{0.1}^x\frac{ \sin(t) }{ t }dt.
\end{equation}

%--------------------------------------------------------------------------------------------------------------------------- 
\subsection{Parametric curve}
%---------------------------------------------------------------------------------------------------------------------------

For the curve
\begin{equation}
    \begin{aligned}
        \gamma\colon \mathopen[ a , b \mathclose]&\to \eR^2 \\
        t&\mapsto (  f_1(t),f_2(t)  ) 
    \end{aligned}
\end{equation}
the syntax is :
\lstinputlisting{sageSnip_5.py}

You can omit the \info{interval} argument; in this case the interval of \info{f1} will be used, but such implicit transfer of property is a bad practice\footnote{\phystricks\ contains lots of such ``if an argument is missing I will search it somewhere'' mechanisms.}.

Here is an example code :
\lstinputlisting{phystricksLARGooSLxQTdPC.py}

The result is on figure \ref{LabelFigLARGooSLxQTdPC}. You see that too few points are plotted, so that the picture is not quite well curved. This problem can be fixed using the \info{plotpoints} attribute of the curve; we will see that later.

\newcommand{\CaptionFigLARGooSLxQTdPC}{This is a parametric curve, a Lyssajou.}
\input{Fig_LARGooSLxQTdPC.pstricks}

%--------------------------------------------------------------------------------------------------------------------------- 
\subsection{Interpolation curve}
%---------------------------------------------------------------------------------------------------------------------------

<++>

%--------------------------------------------------------------------------------------------------------------------------- 
\subsection{Lagrange polynomial, Hermite interpolation}
%---------------------------------------------------------------------------------------------------------------------------

<++>

%--------------------------------------------------------------------------------------------------------------------------- 
\subsection{Compute more plotpoints (sample)}
%---------------------------------------------------------------------------------------------------------------------------

As seen on figure \ref{LabelFigLARGooSLxQTdPC}, the default setting does not compute enough «intermediate» points to produce a visually correct result on some curves.

The easiest way to make the curve more smooth is to increase the \info{plotpoints} attribute; as an example :

\lstinputlisting{sageSnip_6.py}

For fixing the ideas, let's say \info{plotpoints=100}. Then the default behaviour is to consider \( 100\) values \emph{of the parameters} that regularly spaced between its minimum and its maximum. The drawn curve is then the interpolation curve of the corresponding points.

This is not always adapted, and we have two ways to adapt this mechanism to particular cases.
\begin{description}
    \item[Add selected plot points] We can make compute some more points by adding parameters values to the list \info{added\_plotpoints} :
        \lstinputlisting{sageSnip_7.py}
        In this case we will compute \( 102\) points : the usual \( 100\) plus the ones corresponding to the values \( 0.001 \) and \( \pi/5\) of the parameters.

    \item[Force smoothing]
            We can do
        \lstinputlisting{sageSnip_8.py}
        In this case, the \( 100\) interpolation points will be taken regularly spaced with respect to the integral of the curvature. In other words \( \{ x_i \}_{i=1,\ldots, 50}\) are chosen in such a way that
        \begin{equation}
            \int_{x_i}^{x_{i+1}}c(t)dt
        \end{equation}
        is constant with respect to to \( i\).

\end{description}

An example :

\lstinputlisting{phystricksPBFCooVlPiRBpt.py}.

\begin{center}
   \input{Fig_PBFCooVlPiRBpt.pstricks}
\end{center}
Is it better that figure \ref{LabelFigLARGooSLxQTdPC} ? The four angles are for sure smoother. However, the computation of these points at ``regular curvature'' interval can take forever and it is often much faster to simply add thousands of plotpoints.

%--------------------------------------------------------------------------------------------------------------------------- 
\subsection{Derivative, tangent, and other differential geometry}
%---------------------------------------------------------------------------------------------------------------------------

A \info{phyFunctionGraph} object has a method \info{derivative} that returns a \info{phyFunction} of the derivative.

Here is an example code :

\lstinputlisting{FunctionThird.py}

\begin{center}
   \input{Fig_FunctionThird.pstricks}
\end{center}

You also have methods to get the tangent and normal vector. 

% This picture is also the one used in the README.md. 
% Do not change here without changing there.


\lstinputlisting{phystricksVSJOooJXAwbVEt.py}
\begin{center}
   \input{Fig_VSJOooJXAwbVEt.pstricks}
\end{center}


You can grab a list of points regularly spaced on a curve with respect to the arc length.

\lstinputlisting{phystricksGKMEooBcNxcWBt.py}

\begin{center}
   \input{Fig_GKMEooBcNxcWBt.pstricks}
\end{center}

By the way you should note the method \info{getRegularLengthParameters} that return a list of parameters value such that the corresponding points are regularly spaced on the curve (with respect to the arc length). Namely

\lstinputlisting{sageSnip_10.py}

returns a list of parameters such that the arc length between two points is \( 2\).


%+++++++++++++++++++++++++++++++++++++++++++++++++++++++++++++++++++++++++++++++++++++++++++++++++++++++++++++++++++++++++++ 
\section{Perspective}
%+++++++++++++++++++++++++++++++++++++++++++++++++++++++++++++++++++++++++++++++++++++++++++++++++++++++++++++++++++++++++++

You can make your cube opaque (non-transparent) with the method \info{make\_opaque}.


\begin{example}
    
\lstinputlisting{phystricksIllusionNHwEtp.py}

    \begin{center}
        \input{Fig_IllusionNHwEtp.pstricks}
    \end{center}
\end{example}

%+++++++++++++++++++++++++++++++++++++++++++++++++++++++++++++++++++++++++++++++++++++++++++++++++++++++++++++++++++++++++++ 
\section{Figure, subfigure}
%+++++++++++++++++++++++++++++++++++++++++++++++++++++++++++++++++++++++++++++++++++++++++++++++++++++++++++++++++++++++++++

\begin{enumerate}
    \item
        The caption of the figure is not given in the \phystricks\ code, but has to be inserted in the LaTeX document.
    \item 
        On the contrary, the subfigures caption are from the \phystricks\ code
\end{enumerate}
You should use the utility \info{new\_picture.py} to generate the skeleton of your figure.

%+++++++++++++++++++++++++++++++++++++++++++++++++++++++++++++++++++++++++++++++++++++++++++++++++++++++++++++++++++++++++++ 
\section{Put marks on the objects}
%+++++++++++++++++++++++++++++++++++++++++++++++++++++++++++++++++++++++++++++++++++++++++++++++++++++++++++++++++++++++++++

One can always put a mark on an object; the position is by default automatically determined. The general statement is :

\lstinputlisting{sageSnip_9.py}

%--------------------------------------------------------------------------------------------------------------------------- 
\subsection{On angles}
%---------------------------------------------------------------------------------------------------------------------------

You can add a mark inside an angle. The positioning is automatic, and needs two passes.

\begin{example}

    \lstinputlisting{phystricksFBTCooBKTryQ.py}

Picture : \info{FBTCooBKTryQ}
\begin{center}
   \input{Fig_FBTCooBKTryQ.pstricks}
\end{center}

\end{example}


%+++++++++++++++++++++++++++++++++++++++++++++++++++++++++++++++++++++++++++++++++++++++++++++++++++++++++++++++++++++++++++ 
\section{How to get the LaTeX counters ?}
%+++++++++++++++++++++++++++++++++++++++++++++++++++++++++++++++++++++++++++++++++++++++++++++++++++++++++++++++++++++++++++
\label{SECooKVXMooMKJAXV}

We are going to explain one important mechanism in \phystricks\ about its interaction with \LaTeX. For we consider the code

\lstinputlisting{phystricksOnePoint.py}

We compile it in a Sage terminal :

\lstinputlisting{sageSnip_1.py}

If you input now the file \info{Fig\_OnePoint.pstricks} in your \LaTeX\ document, you'll see a beautiful red point with two marks, a \( P\) and a \( Q\).  

\begin{center}
   \input{Fig_OnePoint.pstricks}
\end{center}

However, the marks are badly placed, this is the sense of the warning about the existence of the file \info{LabelFigOnePoint.phystricks.aux}. In fact the file \info{Fig\_OnePoint.pstricks} does not only contains the tikz code for the picture, but also a pure \LaTeX\ code asking latex to write the dimensions of the boxes \( P\) and \( Q\) in an auxiliary file.

Just in order to make is cryptic, these are lines like :
\begin{verbatim}

\makeatletter\@ifundefined{writeOfphystricks}{\newwrite{\writeOfphystricks}}{}\makeatother%
\setlength{\lengthOfhomemokyDOTSagesrcbinsageipython}{\totalheightof{\(P\)}}%
\immediate\write\writeOfphystricks{totalheightof1903839d9021e180dd790c4cc63081c63b2fe6f1:\the\lengthOfhomemokyDOTSagesrcbinsageipython-}
\end{verbatim}

Now you can reenter Sage and recompile the picture :

\lstinputlisting{sageSnip_2.py}

The warning disappeared and now \phystricks\ has read the auxiliary file containing the dimensions of the boxes. The \( P\) and \( Q\) are then now placed taking their \emph{real} dimension into account.

The auxiliary file contains the lines
\begin{verbatim}
totalheightof1903839d9021e180dd790c4cc63081c63b2fe6f1:6.83331pt-
widthof1903839d9021e180dd790c4cc63081c63b2fe6f1:7.80904pt-
totalheightof15a6448f2b408bb6a0dabb437cc671b7beb909fc:8.77776pt-
widthof15a6448f2b408bb6a0dabb437cc671b7beb909fc:7.90555pt-
\end{verbatim}

The box is identified by a hash of its \LaTeX\ code. The reason is that almost(?) any string can be valid \LaTeX\ code\footnote{Thanks to the \info{catcode} mechanism, it seems to me that latex is the most introspective programming language ever.}, so the parsing of this auxiliary file is more or less impossible if the actual \LaTeX\ code is included.

Relaunch \pdfLaTeX\ and you'll see the points correctly placed.

Conclusion : when you add some \LaTeX\ code in your picture, you need one more pass of \pdfLaTeX and \phystricks\ in order to get the marks right.

This mechanism of making \LaTeX\ write values in an auxiliary file is general and any latex internal counters can be accessed in your python code (as Python's \info{float}).

You don't believe ? Here is a picture with the following specifications :
\begin{enumerate}
    \item
        The line slope is the number of the section (here we have \info{\thesection}=\thesection).
\item
The line is drawn from \( x=0\) to \( x=x_{max}\) computed in such a way that \( y_{max}=5\). 
\item
A dilatation in the \( x\)-direction is computed in such a way that the picture has \SI{10}{\centi\meter} length.
\item
    The page number is written just on the top of coordinates \( (x_{max},y_{max})\).
\end{enumerate}

Here is a newpage for reasons that will be explained right on the next page.

\newpage

Obviously this kind of picture has to be recompiled each time we change the containing document, and it can be wrong if the picture happens to be on the top of a page; in this case, \LaTeX\ sees the request for writing the page number on the bottom of the previous page.

\begin{example}
    
\begin{center}
   \input{Fig_RJDEoobOibtkfv.pstricks}
\end{center}

\lstinputlisting{phystricksRJDEoobOibtkfv.py}
\end{example}


The default value for the \info{section} counter is given to avoid division by zero, because zero is the default-default value : the one which is returned at first compilation, when the auxiliary file does not yet contain the value of the counter (there is a bootstrap here).

%+++++++++++++++++++++++++++++++++++++++++++++++++++++++++++++++++++++++++++++++++++++++++++++++++++++++++++++++++++++++++++ 
\section{Axes and grid}
%+++++++++++++++++++++++++++++++++++++++++++++++++++++++++++++++++++++++++++++++++++++++++++++++++++++++++++++++++++++++++++

Adding axes and grid is as simple as

\lstinputlisting{sageSnip_11.py}

Since the grid has to adapt itself to the drawn objects and the axes have to adapt to the grid :
\begin{itemize}
    \item You have to put these lines \emph{after} any other \info{pspict.DrawGraphs} invocation. If not, the result is unpredictable, but is often an error due to a too large bounding box.
    \item You have to invoke the grid before the axes.
\end{itemize}

%+++++++++++++++++++++++++++++++++++++++++++++++++++++++++++++++++++++++++++++++++++++++++++++++++++++++++++++++++++++++++++ 
\section{Known issues}
%+++++++++++++++++++++++++++++++++++++++++++++++++++++++++++++++++++++++++++++++++++++++++++++++++++++++++++++++++++++++++++

There are some known issues.
\begin{enumerate}
    \item
        When performing a dilatation (especially with different \( x\) and \( y\) factors), some objects do not behave correctly. This is the case of the marks on polygon and the coding of a drawing (small bars in order to indicate that two lines have same length).
    \item
        Rotating a whole picture is very poorly tested.
    \item The following pictures in \href{ http://laurent.claessens-donadello.eu/smath.pdf }{ smath } are incorrect :  \info{JSYR},  \info{ZBHL},  \info{KYVA},  \info{DYJN}. Probably due to a bad managing of the dilatation.
\end{enumerate}

Picture : \info{KEQLooOrxtCN}
\begin{center}
   \input{Fig_KEQLooOrxtCN.pstricks}
\end{center}

Picture : \info{FBTCooBKTryQ}
\begin{center}
   \input{Fig_FBTCooBKTryQ.pstricks}
\end{center}

Picture : \info{UOEOooLxhpSC}
\begin{center}
   \input{Fig_UOEOooLxhpSC.pstricks}
\end{center}


Picture : \info{JRCJooPHFcKn}
\begin{center}
   \input{Fig_JRCJooPHFcKn.pstricks}
\end{center}


Picture \info{SFdgHdO}
The result is on figure \ref{LabelFigSFdgHdO}. % From file SFdgHdO
\newcommand{\CaptionFigSFdgHdO}{<+Type your caption here+>}
\input{Fig_SFdgHdO.pstricks}

\clearpage

       Picture \info{exCircleTwo}
   \begin{center}
\input{Fig_exCircleTwo.pstricks}
   \end{center}
\input{Fig_exCircleTwo.comment}


\setcounter{page}{10}
\setcounter{section}{3}
Picture : \info{RJDEoobOibtkfv}
\begin{center}
    Have to be 10 : \thepage\\
    Have to be 3 : \thesection\\
   \input{Fig_RJDEoobOibtkfv.pstricks}
\end{center}
\input{Fig_RJDEoobOibtkfv.comment}



       Picture \info{exCircleTwo}
   \begin{center}
\input{Fig_exCircleTwo.pstricks}
   \end{center}
\input{Fig_exCircleTwo.comment}




Picture \info{OMPAooMbyOIqeA}
\newcommand{\CaptionFigOMPAooMbyOIqeA}{Marks are correct.}
\input{Fig_OMPAooMbyOIqeA.pstricks}

\clearpage


Picture \info{FMLCooxHtqRzUz}
\begin{center}
   \input{Fig_FMLCooxHtqRzUz.pstricks}
\end{center}
   \input{Fig_FMLCooxHtqRzUz.comment}

\clearpage


Picture \info{BHESoofmkTbbZR} : the famous \( f(x)=\sin(1/x)\).
\begin{center}
   \input{Fig_BHESoofmkTbbZR.pstricks}
\end{center}
   \input{Fig_BHESoofmkTbbZR.comment}

Picture \info{PVRFoobvAzpZTq}
\begin{center}
   \input{Fig_PVRFoobvAzpZTq.pstricks}
\end{center}
   \input{Fig_PVRFoobvAzpZTq.comment}

Picture : \info{CNVAooybLqXmVS}
\begin{center}
   \input{Fig_CNVAooybLqXmVS.pstricks}
\end{center}
   \input{Fig_CNVAooybLqXmVS.comment}

For the numbering on the axes with dilatation : \info{QIXEooejrojKjo}
\begin{center}
   \input{Fig_QIXEooejrojKjo.pstricks}
\end{center}
\input{Fig_QIXEooejrojKjo.comment}


Picture : \info{WQVZooAhkdlegv}
\begin{center}
   \input{Fig_WQVZooAhkdlegv.pstricks}
\end{center}
Comment : The angles are red and an arrow is drawn in the trigonometric sense.



About axes and grid : \info{GGHOookMhIxqIK}
\begin{center}
   \input{Fig_GGHOookMhIxqIK.pstricks}
\end{center}
   \input{Fig_GGHOookMhIxqIK.comment}


This is a circle with a tangent vector and a mark :

\begin{center}
    \input{Fig_TRJEooPRoLnEiG.pstricks}
\end{center}
\input{Fig_TRJEooPRoLnEiG.comment}


Picture \info{UREIooqNGBXtHg}
\begin{center}
   \input{Fig_UREIooqNGBXtHg.pstricks}
\end{center}
\input{Fig_UREIooqNGBXtHg.comment}
Picture \info{QIPRoolQCEnZdx}
\begin{center}
   \input{Fig_QIPRoolQCEnZdx.pstricks}
\end{center}
\input{Fig_QIPRoolQCEnZdx.comment}

\clearpage


Picture \info{ASZLoocnIGlRHf}
\begin{center}
   \input{Fig_ASZLoocnIGlRHf.pstricks}
\end{center}
   \input{Fig_ASZLoocnIGlRHf.comment}


\clearpage

Picture \info{YJEDoojDtSeKHQ}
\begin{center}
   \input{Fig_YJEDoojDtSeKHQ.pstricks}
\end{center}
   \input{Fig_YJEDoojDtSeKHQ.comment}


\clearpage

Picture \info{AxesSecond}
\begin{center}
   \input{Fig_AxesSecond.pstricks}
\end{center}
   \input{Fig_AxesSecond.comment}

Picture \info{exCircle}
   \begin{center}
       \input{Fig_exCircle.pstricks}
   \end{center}
       \input{Fig_exCircle.comment}


One red point :

Picture \info{OnePoint}
\begin{center}
   \input{Fig_OnePoint.pstricks}
\end{center}

Picture \info{GridThree}
\begin{center}
\input{Fig_GridThree.pstricks}
\end{center}
\input{Fig_GridThree.comment}

Picture \info{GridTwo}
\begin{center}
\input{Fig_GridTwo.pstricks}
\end{center}
\input{Fig_GridTwo.comment}



Picture \info{MarkOnPoint}
\begin{center}
\input{Fig_MarkOnPoint.pstricks}
\end{center}
\input{Fig_MarkOnPoint.comment}

File \info{QRJOooKZPUoLlF}
\begin{center}
   \input{Fig_QRJOooKZPUoLlF.pstricks}
\end{center}
   \input{Fig_QRJOooKZPUoLlF.comment}

Sudoku grid.

\begin{center}
   \input{Fig_RVKFooDxrqYXAX.pstricks}
\end{center}
   \input{Fig_RVKFooDxrqYXAX.comment}

   \clearpage

Multiple subfigures :

A point with a mark and its bounding box
Picture \info{QRXCooUmnlhkvh}
\Huge

\begin{center}
   \input{Fig_QRXCooUmnlhkvh.pstricks}
\end{center}
\normalsize
\input{Fig_QRXCooUmnlhkvh.comment}

Picture \info{ALAYooKKrRTkCG}
\begin{center}
   \input{Fig_ALAYooKKrRTkCG.pstricks}
\end{center}
   \input{Fig_ALAYooKKrRTkCG.comment}

Picture \info{PJKBooOhGVPkeR}
\begin{center}
   \input{Fig_PJKBooOhGVPkeR.pstricks}
\end{center}
\input{Fig_PJKBooOhGVPkeR.comment}


Picture \info{LVPSoozFTyaeCG}
\begin{center}
\input{Fig_LVPSoozFTyaeCG.pstricks}
\end{center}
\input{Fig_LVPSoozFTyaeCG.comment}

Picture \info{CUZFooGqZLaAEp}
\begin{center}
   \input{Fig_CUZFooGqZLaAEp.pstricks}
\end{center}
\input{Fig_CUZFooGqZLaAEp.comment}

Picture \info{JSYWooQYduLVLS}
\begin{center}
   \input{Fig_JSYWooQYduLVLS.pstricks}
\end{center}
\input{Fig_JSYWooQYduLVLS.comment}

Picture \info{LWVXooPyIlOKNd}
\begin{center}
   \input{Fig_LWVXooPyIlOKNd.pstricks}
\end{center}
\input{Fig_LWVXooPyIlOKNd.comment}

Picture : \info{TKXZooLwXzjS}
\newcommand{\CaptionFigTKXZooLwXzjS}{Le champ de vecteurs $F(x,y)=\frac{1}{ x }(1,0)$.}
\begin{center}
   \input{Fig_TKXZooLwXzjS.pstricks}
\end{center}
   \input{Fig_TKXZooLwXzjS.comment}

Picture : \info{SMXRooCnrlNw}
\begin{center}
   \input{Fig_SMXRooCnrlNw.pstricks}
\end{center}
   \input{Fig_SMXRooCnrlNw.comment}


Picture : \info{EDEYRhQ}
\begin{center}
   \input{Fig_EDEYRhQ.pstricks}
\end{center}
   \input{Fig_EDEYRhQ.comment}

Picture : \info{VAAYooXndWQq}
\begin{center}
   \input{Fig_VAAYooXndWQq.pstricks}
\end{center}



Picture : TgCercleTrigono
\begin{center}
   \input{Fig_TgCercleTrigono.pstricks}
\end{center}

Picture : Refraction
\begin{center}
   \input{Fig_Refraction.pstricks}
\end{center}

Picture \info{CercleTrigono}
The result is on figure \ref{LabelFigCercleTrigono}. % From file CercleTrigono
\newcommand{\CaptionFigCercleTrigono}{<+Type your caption here+>}
\input{Fig_CercleTrigono.pstricks}

Picture \info{CWKJooppMsZXjw}
\begin{center}
   \input{Fig_CWKJooppMsZXjw.pstricks}
\end{center}

Picture \info{AMDUooZZUOqa}
The result is on figure \ref{LabelFigAMDUooZZUOqa}. % From file AMDUooZZUOqa
\newcommand{\CaptionFigAMDUooZZUOqa}{<+Type your caption here+>}
\input{Fig_AMDUooZZUOqa.pstricks}




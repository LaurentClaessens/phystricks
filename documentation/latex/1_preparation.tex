%+++++++++++++++++++++++++++++++++++++++++++++++++++++++++++++++++++++++++++++++++++++++++++++++++++++++++++++++++++++++++++ 
\section{Preparation}
%+++++++++++++++++++++++++++++++++++++++++++++++++++++++++++++++++++++++++++++++++++++++++++++++++++++++++++++++++++++++++++

%--------------------------------------------------------------------------------------------------------------------------- 
\subsection{Dependencies and installation}
%---------------------------------------------------------------------------------------------------------------------------

\begin{enumerate}
    \item
        You need a working \href{ http://sagemath.org }{ sage } installation.
    \item
        Download \phystricks from \href{ https://github.com/LaurentClaessens/phystricks }{ github } and make it available from Sage (\info{from phystricks import *} has to work).
    \item
        I don't even speak about having a working \LaTeX\ installation with Tikz installed.
\end{enumerate}

%--------------------------------------------------------------------------------------------------------------------------- 
\subsection{In your \LaTeX\ file}
%---------------------------------------------------------------------------------------------------------------------------

The preamble of your \LaTeX\ file has to contain

\begin{verbatim}
    \usepackage{calc}   
    \usepackage{tikz}
    \usetikzlibrary{patterns}
    \usetikzlibrary{calc}
    \newcounter{defHatch}
    \newcounter{defPattern}
    \setcounter{defHatch}{0}
    \setcounter{defPattern}{0}
\end{verbatim}

and you (don't really) have to compile with \info{pdflatex -shell-escape}.

%--------------------------------------------------------------------------------------------------------------------------- 
\subsection{Structure of your \phystricks\ file}
%---------------------------------------------------------------------------------------------------------------------------

Most of your \phystricks files will have the following structure :

\lstinputlisting{phystricksQLXFooBDalHMaT.py}

We will see later the significance of these lines.

%+++++++++++++++++++++++++++++++++++++++++++++++++++++++++++++++++++++++++++++++++++++++++++++++++++++++++++++++++++++++++++
\section{Draw points}
%+++++++++++++++++++++++++++++++++++++++++++++++++++++++++++++++++++++++++++++++++++++++++++++++++++++++++++++++++++++++++++

Here is the code corresponding to one red point with two marks.

\lstinputlisting{phystricksOnePoint.py}

\begin{enumerate}
    \item
        
Compile it once in the Sage terminal :

\lstinputlisting{sageSnip_1.py}

\item


\end{enumerate}
<++>


If you want to know why, this is related to the mechanism of catching the \LaTeX's internal counters(here the size of the box) by \phystricks, see section \ref{SECooKVXMooMKJAXV}.

%+++++++++++++++++++++++++++++++++++++++++++++++++++++++++++++++++++++++++++++++++++++++++++++++++++++++++++++++++++++++++++ 
\section{How to get the LaTeX counters ?}
%+++++++++++++++++++++++++++++++++++++++++++++++++++++++++++++++++++++++++++++++++++++++++++++++++++++++++++++++++++++++++++
\label{SECooKVXMooMKJAXV}

We are going to explain one important mechanism in \phystricks\ about its interaction with \LaTeX. For we consider the code

\lstinputlisting{phystricksOnePoint.py}

We compile it in a Sage terminal :

\lstinputlisting{sageSnip_1.py}

If you input now the file \info{Fig\_OnePoint.pstricks} in your \LaTeX\ document, you'll see a beautiful red point with two marks, a \( P\) and a \( Q\).  

\begin{center}
   \input{Fig_OnePoint.pstricks}
\end{center}

However, the marks are badly placed, this is the sense of the warning about the existence of the file \info{LabelFigOnePoint.phystricks.aux}. In fact the file \info{Fig\_OnePoint.pstricks} does not only contains the tikz code for the picture, but also a pure \LaTeX\ code asking latex to write the dimensions of the boxes \( P\) and \( Q\) in an auxiliary file.

Just in order to make is cryptic, these are lines like :
\begin{verbatim}

\makeatletter\@ifundefined{writeOfphystricks}{\newwrite{\writeOfphystricks}}{}\makeatother%
\setlength{\lengthOfhomemokyDOTSagesrcbinsageipython}{\totalheightof{\(P\)}}%
\immediate\write\writeOfphystricks{totalheightof1903839d9021e180dd790c4cc63081c63b2fe6f1:\the\lengthOfhomemokyDOTSagesrcbinsageipython-}
\end{verbatim}

Now you can reenter Sage and recompile the picture :

\lstinputlisting{sageSnip_2.py}

The warning disappeared and now \phystricks\ has read the auxiliary file containing the dimensions of the boxes. The \( P\) and \( Q\) are then now placed taking their \emph{real} dimension into account.

The auxiliary file contains the lines
\begin{verbatim}
totalheightof1903839d9021e180dd790c4cc63081c63b2fe6f1:6.83331pt-
widthof1903839d9021e180dd790c4cc63081c63b2fe6f1:7.80904pt-
totalheightof15a6448f2b408bb6a0dabb437cc671b7beb909fc:8.77776pt-
widthof15a6448f2b408bb6a0dabb437cc671b7beb909fc:7.90555pt-
\end{verbatim}

The box is identified by a hash of its \LaTeX\ code. The reason is that almost(?) any string can be valid \LaTeX\ code\footnote{Thanks to the \info{catcode} mechanism, it seems to me that latex is the most introspective programming language ever.}, so the parsing of this auxiliary file is more or less impossible if the actual \LaTeX\ code is included.

Relaunch \pdfLaTeX\ and you'll see the points correctly placed.

Conclusion : when you add some \LaTeX\ code in your picture, you need one more pass of \pdfLaTeX and \phystricks\ in order to get the marks right.

This mechanism of making \LaTeX\ write values in an auxiliary file is general and any latex internal counters can be accessed in your python code (as Python's \info{float}).

You don't believe ? Here is a line whose slope is the number of the section (here we have \info{\\thesection}=\thesection), drawn from \( x=0\) to \( x=x_{max}\) computed in such a way that \( y_{max}=5\). A dilatation is computed in such a way that the picture has \SI{10}{\centi\meter} length.

The page number is also written. 

\begin{center}
   \input{Fig_RJDEoobOibtkfv.pstricks}
\end{center}

Obviously this kind of picture has to be recompiled each time we change the containing document.

Here is the code :

\lstinputlisting{phystricksRJDEoobOibtkfv.py}

\documentclass[a4paper,12pt]{article}

\usepackage{latexsym}
\usepackage{amsfonts}
\usepackage{amsmath}
\usepackage{amsthm}
\usepackage{amssymb}
\usepackage{bbm}			% Pour les symboles des ensembles N,Z,Q,H,R,C; il remplace \usepackage{mathbbol} et les \def\eR{\mathbb{R}}

\usepackage{pstricks,pst-eucl,pstricks-add,graphicx}	% packages utilised by the generated pstricks code.
\usepackage{graphicx}

\usepackage{fancyvrb}
\usepackage{multicol}

\usepackage[ps2pdf]{hyperref}
\hypersetup{colorlinks=true,linkcolor=blue}

%%%%%%%%%%%%%%%%%%%%%%%%%%
%
%   Les lignes magiques pour le texte en anglais.
%
%%%%%%%%%%%%%%%%%%%%%%%%


\usepackage[a4paper]{geometry}
\usepackage{textcomp}
%\usepackage{mathpazo}
\usepackage{lmodern}
\usepackage[utf8]{inputenc}
\usepackage[T1]{fontenc}


%%%%%%%%%%%%%%%%%%%%%%%%%%
%
%   Numérotations
%
%%%%%%%%%%%%%%%%%%%%%%%%

\newcounter{numtho}

\theoremstyle{definition}
   \newtheorem{exercise}{Exercise}			% Les exercices ne se numérotent pas avec les autres, pour que les références soient plus faciles à suivre dans la partie corrigée.
\theoremstyle{remark} 
   \newtheorem{exemple}[numtho]{Example}
   \newtheorem{remarque}[numtho]{Remark}
   \newtheorem{erreur}[numtho]{Error}
   \newtheorem{probleme}[numtho]{\fbox{\bf Probl\`emes et choses \`a faire}}

\theoremstyle{plain}
   \newtheorem{theoreme}[numtho]{Theorem}
   \newtheorem{proposition}[numtho]{Proposition}
   \newtheorem{definition}[numtho]{Définition}
   \newtheorem{lemme}[numtho]{Lemma}


%\renewcommand{\theenumi}{\alph{enumi}}
%\renewcommand{\labelenumi}{{(\roman{enumi})}}

%%%%%%%%%%%%%%%%%%%%%%%%%%
%
%   Les ensembles
%
%%%%%%%%%%%%%%%%%%%%%%%%

\def\eA{\mathbbm{A}}
\def\eC{\mathbbm{C}}
\def\eH{\mathbbm{H}}
\def\eK{\mathbbm{K}}  % Les ensembles de nombres
\def\eN{\mathbbm{N}}
\def\eQ{\mathbbm{Q}}
\def\eR{\mathbbm{R}}
\def\eZ{\mathbbm{Z}}

%\DeclareMathOperator{\sinh}{sinh}


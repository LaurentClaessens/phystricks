%+++++++++++++++++++++++++++++++++++++++++++++++++++++++++++++++++++++++++++++++++++++++++++++++++++++++++++++++++++++++++++
\section{Draw points}
%+++++++++++++++++++++++++++++++++++++++++++++++++++++++++++++++++++++++++++++++++++++++++++++++++++++++++++++++++++++++++++

%---------------------------------------------------------------------------------------------------------------------------
\subsection{Hello world!}
%---------------------------------------------------------------------------------------------------------------------------


The first step is to draw a point. More precisely, we are going to create figure which contains a pspicture of a point. For that purpose, the phystricks module provide the class \verb+phystricks.Point+ and the method \verb+DrawGraph+ on the class \verb+phystricks.pspicture+. 

Here is the code to be written in you python's script :
\VerbatimInput[tabsize=3]{first_example.py}
And here are the lines to be included in your \LaTeX\ file :
\begin{verbatim}
The result is given on the figure \ref{LabelFigOnePoint}
\newcommand{\CaptionFigOnePoint}{This is my first point}
\input{Fig_OnePoint.pstricks}
\end{verbatim}
The result is given on the figure \ref{LabelFigOnePoint}
\newcommand{\CaptionFigOnePoint}{This is my first point}
\input{Fig_OnePoint.pstricks}

If the point has a name, we can mark it with the method
\begin{verbatim}
P.mark(dist,angle,mark)
\end{verbatim}
where \verb+dist+ is the distance between the point and the mark, \verb+angle+ is the angle and \verb+mark+ is the text to be placed. This is typically a \LaTeX code like \verb+$P$+ if $P$ is the name of the point. The parameters are inspired from the command \verb+\pstGeonode+ from the package \verb+Euclide+:
\begin{enumerate}

	\item
		Distance between the point and the mark,
	\item
		the angle
	\item
		the mark, to be interpreted by \LaTeX.

\end{enumerate}

We change the color and the symbol of the point using the attributes
\begin{verbatim}
P.parameters.color
P.parameters.symbol
\end{verbatim}
The symbol is by default \verb+"*"+, so that a circular point appears, while the default color is \verb+"black"+. The available symbols are\footnote{This list is taken from the documentation of \texttt{pst-eucl} without the authorization of the author, and constitutes a possible violation of the \LaTeX{} Project Public Licence.}
\label{PgTableauMarques}
\begin{multicols}{3}
  \begin{itemize}\psset{dotscale=2}
  \item \texttt{*} : \psdots(.5ex,.5ex)
  \item \texttt{o} : \psdots[dotstyle=o](.5ex,.5ex)
  \item \texttt{+} : \psdots[dotstyle=+](.5ex,.5ex)
  \item \texttt{x} : \psdots[dotstyle=x](.5ex,.5ex)
  \item \texttt{asterisk} : \psdots[dotstyle=asterisk](.5ex,.5ex)
  \item \texttt{oplus} : \psdots[dotstyle=oplus](.5ex,.5ex)
  \item \texttt{otimes} : \psdots[dotstyle=otimes](.5ex,.5ex)
  \item \texttt{triangle} : \psdots[dotstyle=triangle](.5ex,.5ex)
  \item \texttt{triangle*} : \psdots[dotstyle=triangle*](.5ex,.5ex)
  \item \texttt{square} : \psdots[dotstyle=square](.5ex,.5ex)
  \item \texttt{square*} : \psdots[dotstyle=square*](.5ex,.5ex)
  \item \texttt{diamond} : \psdots[dotstyle=diamond](.5ex,.5ex)
  \item \texttt{diamond*} : \psdots[dotstyle=diamond*](.5ex,.5ex)
  \item \texttt{pentagon} : \psdots[dotstyle=pentagon](.5ex,.5ex)
  \item \texttt{pentagon}* : \psdots[dotstyle=pentagon*](.5ex,.5ex)
  \item \texttt{|} : \psdots[dotstyle=|](.5ex,.5ex)
  \end{itemize}
\end{multicols}
If you want to not print the point, you have to set the symbol to \texttt{none}. That can be useful in order to impose the size of the bounding box.

These customisations are exemplified in the following code:
\VerbatimInput[tabsize=3]{MarkOnPoint.py}
The result is on figure \ref{LabelFigMarkOnPoint}.
\newcommand{\CaptionFigMarkOnPoint}{A point with a mark. Notice that one can mark the point with any string, including \LaTeX\ code.}
\input{Fig_MarkOnPoint.pstricks}
%---------------------------------------------------------------------------------------------------------------------------
\subsection{Draw several points at once}
%---------------------------------------------------------------------------------------------------------------------------

You know that python is a programming language\footnote{The author is used to troll against \LaTeX\ as programming language.}. The ultimate purpose of \verb+phystricks+ is to allow the use of that powerful programming language in order to create figures.

The following code draw the first $10$ points of the sequence $x_n=\frac{ (-1)^n }{ n }$.
\VerbatimInput[tabsize=2]{Sequence.py}
The result is on the figure \ref{LabelFigsequence}.
\newcommand{\CaptionFigsequence}{The first points of $x_n=(-1)^n/n$.}
\input{Fig_sequence.pstricks}

Notice that the marks $P_i$ are not taken into account in the bounding box, because it is quite impossible for python to know the size of a given \LaTeX expression. A solution would be to compile the expression in an intermediate \LaTeX file and extract the size of the box. I'm not going to implement that \ldots A consequence of it is that the export to \verb+pdf+ is bad in the case of this figure: the marks are cut.


%---------------------------------------------------------------------------------------------------------------------------
\subsection{Points in polar coordinates}
%---------------------------------------------------------------------------------------------------------------------------

A point in polar coordinates is created by \verb+PolarPoint(r,theta)+. It returns an instance of the class \verb+Point+, so that
\begin{verbatim}
print PolarPoint(1,45).coordinates()
\end{verbatim}
will still print the Cartesian coordinates of that point.

